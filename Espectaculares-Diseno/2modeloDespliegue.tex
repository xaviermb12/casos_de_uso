%=========================================================
\chapter{Modelo de Despliegue}
\label{cap:despliegue}

Este capítulo contiene las instrucciones de instalación de dependencias del proyecto, tanto para el proyecto en Desarrollo como la resolución de dependencias en Producción.

Contiene también los Lineamientos / Buenas Prácticas de Diseño, las cuáles describen la forma en que se escribe el código y cómo se manejan la excepciones en caso de ocurrir un error. \\

%---------------------------------------------------------
\section{Entorno de Producción}

Requisito importante para Desarrollo y Produccion: Tener/rentar un servidor con S.O. Fedora y permisos de Root. NOTA: La forma de instalación de Desarrollo y Produccion es prácticamente la misma, exceptuando por el manejador de Bases de Datos \\

\subsection{Instalar Apache en el Servidor}
\begin{enumerate}
    \item Para instalar Apache, se actualiza e instala usando dnf.
    \begin{lstlisting} 
    # sudo dnf install httpd
    \end{lstlisting}
    \item Habilitar el servicio httpd e iniciar Apache.
    \begin{lstlisting} 
    # sudo systemctl enable httpd.service
    # sudo systemctl start httpd.service
    \end{lstlisting}
    \item Comprobar la versión de Apache
    \begin{lstlisting}
    # httpd -v
    \end{lstlisting}
    \item Instalar y activar el firewall
    \begin{lstlisting}
    # sudo dnf install firewalld -y
    # sudo systemctl start firewalld
    # sudo systemctl enable firewalld
    \end{lstlisting}
    \item Habilitar el acceso a http o https a través del cortafuegos del sistema
    \begin{lstlisting}
    # sudo firewall-cmd --add-service={http,https} --permanent
    # sudo firewall-cmd --reload
    \end{lstlisting}
    \item Comprobar el estado del servicio httpd
    \begin{lstlisting}
    # sudo systemctl status httpd.service
    \end{lstlisting}
    \item Probar el servidor web Apache yendo en el navegador a: http://Apache-Server-IP
\end{enumerate}

\subsection{Instalar SQL Server}
\begin{enumerate}
    \item Descargar el archivo de Configuración del Repositorio de Microsoft
    \begin{lstlisting}
    # sudo curl -o /etc/yum.repos.d/mssql-server.repo \
        https://packages.microsoft.com/config/rhel/7/mssql-server-2017.repo
    \end{lstlisting}
    \item Instalar SQL Server 
    \begin{lstlisting}
    # sudo yum install -y mssql-server
    \end{lstlisting}
    \item Correr el siguiente comando y seguir los pasos para terminar la configuración
    \begin{lstlisting}
    # sudo /opt/mssql/bin/mssql-conf setup
    \end{lstlisting}
    \item Verificar que el servicio esté corriendo
    \begin{lstlisting}
    # systemctl status mssql-server
    \end{lstlisting}
    \item Permitir conexiones remotas a través del Firewall
    \begin{lstlisting}
    # sudo firewall-cmd --zone=public --add-port=1433/tcp --permanent
    # sudo firewall-cmd --reload
    \end{lstlisting}
\end{enumerate}

\subsection{Instalar PHP}
\begin{enumerate}
    \item Instalar PHP
    \begin{lstlisting}
    # sudo dnf install php php-common
    \end{lstlisting}
    \item Instalar algunos módulos usados comúnmente con PHP y reiniciamos Apache para que los cambios surtan efecto
    \begin{lstlisting}
    # sudo dnf install php php-common php-mysqlnd php-gd php-imap php-xml 
        php-cli php-opcache php-mbstring
    # systemctl restart httpd
    \end{lstlisting}
    \item Comprobamos la versión de PHP
    \begin{lstlisting}
    # php -v
    \end{lstlisting}
\end{enumerate}

%---------------------------------------------------------
\section{Entorno de Desarrollo}

Requisito importante: Tener un servidor con S.O. Fedora instalado y permisos de Root. \\

\subsection{Instalar Apache en el Servidor}
\begin{enumerate}
    \item Para instalar Apache, se actualiza e instala usando dnf.
    \begin{lstlisting} 
    # sudo dnf install httpd
    \end{lstlisting}
    \item Habilitar el servicio httpd e iniciar Apache.
    \begin{lstlisting} 
    # sudo systemctl enable httpd.service
    # sudo systemctl start httpd.service
    \end{lstlisting}
    \item Comprobar la versión de Apache
    \begin{lstlisting}
    # httpd -v
    \end{lstlisting}
    \item Instalar y activar el firewall
    \begin{lstlisting}
    # sudo dnf install firewalld -y
    # sudo systemctl start firewalld
    # sudo systemctl enable firewalld
    \end{lstlisting}
    \item Habilitar el acceso a http o https a través del cortafuegos del sistema
    \begin{lstlisting}
    # sudo firewall-cmd --add-service={http,https} --permanent
    # sudo firewall-cmd --reload
    \end{lstlisting}
    \item Comprobar el estado del servicio httpd
    \begin{lstlisting}
    # sudo systemctl status httpd.service
    \end{lstlisting}
    \item Probar el servidor web Apache yendo en el navegador a: http://Apache-Server-IP
\end{enumerate}

\subsection{Instalar MariaDB}
\begin{enumerate}
    \item Instalar la ultima version de MariaDB con el siguiente comando
    \begin{lstlisting}
    # sudo dnf install mariadb-server
    \end{lstlisting}
    \item Checar la version de instalación de MariaDB
    \begin{lstlisting}
    # mysql -V
    \end{lstlisting}
    \item Iniciar / habilitar el servicio mariaDB y comprobar su estado.
    \begin{lstlisting}
    # sudo systemctl start mariadb.service
    # sudo systemctl enable mariadb.service
    # sudo systemctl status mariadb.service
    \end{lstlisting}
\end{enumerate}

\subsection{Instalar PHP}
\begin{enumerate}
    \item Instalar PHP
    \begin{lstlisting}
    # sudo dnf install php php-common
    \end{lstlisting}
    \item Instalar algunos módulos usados comúnmente con PHP y reiniciamos Apache para que los cambios surtan efecto
    \begin{lstlisting}
    # sudo dnf install php php-common php-mysqlnd php-gd php-imap php-xml 
        php-cli php-opcache php-mbstring
    # systemctl restart httpd
    \end{lstlisting}
    \item Comprobamos la versión de PHP
    \begin{lstlisting}
    # php -v
    \end{lstlisting}
\end{enumerate}

\section{Lineamientos de Diseño}

\cdtInstrucciones{
    Cómo se deberá escribir cada parte del código(reglas). Cómo se manejaran las Excepciones(reglas generales, descripción de la excepciones)
}

En esta sección describiremos las buenas prácticas de programación a seguir durante el proyecto, y que por ser nuestro estándar deberán contener todos los archivos de código en él contenidos. \\
Previniendo también los fallos del sistema, se describirán los posibles errores/excepciones y la forma en la cual los manejaremos. \\

\subsection{Escritura del Código}
	\subsubsection{General}
		.\\
		\textit{Regla 1.-} El código debe estar correctamente indentado para facilitar su lectura. \\
		\textit{Regla 2.-} Los nombres(tanto de métodos como de variables) deben describir para lo que siven, para facilitar de nuevo la lectura del código y que otro desarrollador pueda continuar con la construcción de otro módulo sin problemas. \\
		\textit{Regla 3.-} Debe reducirse cuanto sea posible el uso de variables globales(Con una excepción, que mencionaremos más adelante), con el objetivo de hacer los módulos independientes y menos susceptibles a los errores. \\
		\textit{Regla 4.-} Los archivos se modularizarán, procurando que cada uno no tenga más de 300 líneas de código, de nuevo, para facilitar su lectura y modificación. \\
		\textit{Regla 5.-} Todo lo que tenga que ver con envío a frontend se minificará en producción, para disminuir el peso del archivo y aumentar la velocidad de carga, así como desaparecer comentarios para evitar fugas de información que no debería ser visible para los usuarios finales. \\
		\textit{Regla 6.-}  Se usarán variables para indicar que alguien es desarrollador, y así facilitar las pruebas del sistema. Todo lo que tenga que ver con testing desaparecerá si dichas variables no están activas, y se limitará su uso en producción.

	\subsubsection{PHP}
	.\\
	\textit{Regla 1.-} No está permitido imprimir variables con PHP dentro del HTML, al estilo
	\begin{lstlisting}
	<input type="text" value="<?php echo 'variable desde php';?>">
	\end{lstlisting}
	Debido a que le quita legibilidad al código y, hace la aplicación menos escalable. \\
	\textit{Regla 2.-} Se marcará claramente la funcionalidad y uso de los archivos en API vs los archivos de la vista. \\
	\textit{Regla 3.-} Los archivos PHP que contengan HTML se deberán modularizar de forma que se puedan mandar llamar desde otras vistas sin problemas. \\
	\textit{Regla 4.-} Debido a que al hacer varias inclusiones de archivos en sub-archivos consecutivamente, si están en direcciones diferentes da problemas, se crearán funciones para poder incluir archivos desde la raíz del proyecto, o el punto desde donde queramos incluir, apoyándonos en la variable \$\_SERVER['SERVER\_ADDR'], que contiene la raíz del servidor.
		
	\subsubsection{JavaScript}
	.\\
	\textit{Regla 1.-} Para evitar cuellos de botella al hacer peticiones, haremos uso de las promesas y el estándar Fetch, usado en las últimas versiones de ECMAscript. \\

	\subsubsection{CSS}
	Aunque varias de las siguientes reglas no son posibles con CSS normal, con el uso de SCSS se nos abren varias posibilidades, como:\\
	\textit{Regla 1.-} El uso de variables globales de CSS para permitir hacer un cambio de colores generales cambiando sólo la variable y no tener que modificar todos los lugares donde aparecen. \\
	\textit{Regla 2.-} Aunque los estilos sean generales, se modularizarán los archivos fuente, donde cada uno afectará la parte de la pantalla a la que le toque afectar. \\
	\textit{Regla 3.-} Se evitará en lo posible el anidamiento de componentes en CSS, para evitar la dependencia de sus padres, por lo que se fomentará el uso de clases para reutilizar las mismas.
	
	
\subsection{Excepciones}
	\subsubsection{Descripción de Excepciones}
	%   Solo son ideas, no es el completo
	.\\
	\textit{Excepcion 1.-No\_Encontrado. } Excepción lanzada en caso de que el usuario solicite un recurso inexistente en nuestro servidor. \\  
	\textit{Excepcion 2.-Sin\_Permisos} Excepción que se lanzará cuando alguie con acceso al sistema solicite via URL un recurso al que no tenga permiso para acceder. \\
	\textit{Excepcion 3.-Consulta\_Vacía} Excepción que se lanzará cuando una consulta no tenga ningún resultado coincidente. \\
	\textit{Excepcion 2.-No\_JSON} Excepción que se lanzará al intentar convertir un json, cuando la petición recibida no tenga dicho formato. \\
	\textit{Excepcion 2.-No\_Objeto} Excepción lanzada cuando se espere un objeto y se reciba otra cosa. \\
	\textit{Excepcion 2.-Variable\_Inexistente} Excepción lanzada cuando una variable no exista o no esté seteada. \\
	\textit{Excepcion 2.-Variable\_Nula} Excepción lanzada cuando se requieran hacer operaciones con una variable y ésta tenga valor nulo. \\
	\textit{Excepcion 2.-Accion\_Inexistente} Excepción lanzada cuando la acción que estemos intentando llamar no exista. \\
	\textit{Excepcion 2.-Extension\_No\_Aceptada} Excepción lanzada cuando se intente subir un archivo con una extensión que no aceptamos. \\
	\textit{Excepcion 2.-Extension\_Falsa} Excepción lanzada cuando se intente subir un archivo con cierta extensión, pero que sus headers contengan otro tipo de archivo. \\
    
%    \subsubsection{Manejo de Excepciones}
