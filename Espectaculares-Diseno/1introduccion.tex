%=========================================================
\chapter{Introducción}

    Este documento contiene el Diseño del proyecto ``Gestión de Espectaculares`` correspondiente al trabajo realizado en el semestre 2019-1 para la materia de Análisis y Diseño Orientado a Objetos en el grupo 2CV9 por el equipo ``{\em Tlatotech Company}``

%---------------------------------------------------------
\section{Presentación}

    Este documento contiene el diseño del proyecto ``Gestion de espectaculares``, detallando Arquitectura utilizada, dependencias de Software, formas de acceso a los datos y distribución de los módulos involucrados. Tiene como objetivo ser el plano a seguir por el equipo de trabajo para construir el sistema. Debe ser aprobado por los principales responsables del proyecto para poder proceder a su construcción.
	
	Este documento es el C1-PP1 del proyecto ``{\em Gestión de Espectaculares}''.
	
%---------------------------------------------------------
\section{Organización del contenido}

	En el capítulo \ref{cap:despliegue} se presentan las instrucciones para instalar lo que requiere el proyecto para funcionar(sus dependencias) tanto en el entorno de Desarrollo como en Produccion. También se abarcan los lineamientos de diseño, así como el manejo de excepciones y errores de manejo.
	
	En el capítulo \ref{cap:modeloEstatico} se presenta el modelado del sistema sin tener en cuenta la acción del tiempo sobre él, y la distribución de subsistemas, paquetes y clases.
	
	% ------------> No se incluye el archivo 4modeloDinamico.tex dentro del main
	En el capítulo \ref{cap:modeloDinamico} se muestra el modelado del sistema, tomando en cuenta la acción del tiempo. Este capítulo complementa lo visto en el capítulo \ref{cap:modeloEstatico}.
	
	En el capítulo \ref{cap:persistencia} abordamos la forma en que se persistirán y se organizarán los datos dentro del sistema, así como la forma en que accederemos a ellos.

%---------------------------------------------------------


