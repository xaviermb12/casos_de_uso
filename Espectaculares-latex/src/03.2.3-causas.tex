\subsubsection{Causas}

\begin{enumerate}
	\item Se lleva un control mediante papeles y con pocas personas con acceso a los mismos, que demoran cierto tiempo para responder las solicitudes de información de otros empleados, añadiendo el que si se encuentran saturadas de trabajo o indispuestas no pueden procesar las solicitudes durante cierto tiempo.
	\item Para obtener información sobre cierto espectacular, el cliente necesita llamar a la empresa para preguntar directamente a un vendedor, limitándolo a lo que el vendedor, con base en sus criterios pueda ofrecerle. Añadiéndole el lento acceso a la información debido al papeleo.
	\item Falta de clasificacion efectiva de espectaculares, así como que la ubicación mediante nombre de calle, número y datos por el estilo podrían darle una perspectiva errónea al cliente de donde se encuentra el espectacular y cómo es.
	\item Ausencia de una forma de recordar a los empleados cuándo son necesarias varias acciones, tales como mantenimiento, renovación de permisos y seguros, de instalación de espectaculares, impresión de diseños o chequeo de pagos del cliente.
	\item Ausencia de acceso ágil al conocimiento de si los permisos, servicios y seguro de cierto espectacular se encuentran activos, por vencer o vencidos.
	\item Necesidad de cálculos a mano por parte del vendedor para realizar presupuestos a los clientes.
	\item Falta de notificación a encargados y responsables de asignación de tareas a tiempo, así como información de si fué cumplida o no dicha tarea.
\end{enumerate}