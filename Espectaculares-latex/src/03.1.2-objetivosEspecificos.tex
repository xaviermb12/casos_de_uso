\subsubsection{Objetivos Específicos}

\begin{itemize}
	\item Mostrar al cliente una interfaz llamativa y fácil de usar, donde pueda hallar y filtrar los espectaculares que más le interesen, mediante filtros como "disponible", "en mantenimiento", "ubicación" y "precio".
    
	\item Facilitar al usuario el manejo y acceso de documentos legales y permisos, así como el status actual de dichos documentos.
    
	\item Llevar un registro y administración de las fechas en las cuales debe darse mantenimiento a los espectaculares, así como dar la posibilidad de gestionar al equipo y notificar las acciones que les conciernen.
    
	\item Ofrecer al cliente la posibilidad de elegir entre una serie de diseños o subir el suyo propio, así como una gama de tipos y tamaños de espectaculares.
    
	\item Hacer identificación entre los roles de los usuarios del sistema, permitiendo a cada uno acceder únicamente a las partes que le correspondan.
    
	\item Llevar el registro de los servicios/recursos que se pagan por cada espectacular.
    
	\item Manejar la relación entre espectaculares y clientes mediante contratos, especificando acciones a tomar a partir de dichos contratos y permitiendo reutilizarlos o modificar algunas cláusulas en casos especiales.
    
	\item Facilitar la comunicación de quejas o reporte de daños a los encargados correspondientes, y con los dueños de la infraestructura.
    
	\item Utilizar datos bancarios y números de depósito en la plataforma, en vez de realizar manejo de dinero dentro de la misma, para mayor seguridad.
    
	\item Llevar la planeación del ciclo de vida de los contratos de espectaculares, fijando fechas para las cuales deben haber ocurrido acciones de parte de la empresa o el cliente, estableciendo acciones a tomar en caso de que dichas acciones no hayan siso cumplidas, y permitiendo una mayor precisión en la gestión de tiempo de todo el proceso.
    
	\item Asignar, dependiendo del rol, a empleados a ciertas labores que les correspondan en cierto tiempo. En caso del empleado asignado, llevar el registro de las tareas que cumplió a tiempo y las que no.
    
	\item Localizar, dependiendo de la zona en la que está el espectacular, el tipo de público al que causará mayor impacto, y clasificarlos o tomar decisiones basándose en dicha información.
    
	\item Realizar estadísticas o listado, según sea el caso, de los puntos antes mencionados.
\end{itemize}
